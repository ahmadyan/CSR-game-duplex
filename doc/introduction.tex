\section{Introduction}
In recent years, there has been a wide range of studies on the role of social and distributed networks in various disciplinary areas such as economics, computer science, epidemiology, and engineering. In particular, availability of large data from online social networks and advances in control of distributed systems have drawn the attention of many researchers to model various phenomena in social and distributed networks using some mathematical tools such as game theory in order to exploit some of the hidden properties of such networks. Such studies not only improve our understanding of the complex nature of social and distributed events, but also enable us to devise more efficient algorithms toward some desired outcomes in such networks.

Due to accelerated growth of economic networks and advances in using game-theoretic tools in various applications, modeling of distributed network storage has become an important issue. In general, distributed network storage games or resource allocation games are characterized by a set of agents who compete for the same set of resources \cite{pacifici2012convergence,masucci2014strategic}, and arise in a wide variety of contexts such as congestion games \cite{milchtaich1996congestion,ackermann2008impact,fabrikant2004complexity}, load balancing \cite{ghosh1994dynamic}, peer-to-peer systems \cite{pollatos2008social}, web-caches \cite{gopalakrishnan2012cache}, content management \cite{pollatos2008social}, and market sharing games \cite{goemans2006market}. Among many problems that arise in such a context, one that stands out is distributed replication, which not only improves the availability of resources for users, but also increases the reliability of the entire network with respect to customer requests \cite{chun2004selfish}, \cite{goyal2000learning}.

Distributed replication games with servers that have access to all the resources and are accessible at some cost by users have been studied in \cite{laoutaris2006distributed}. Moreover, the uncapacitated selfish replication game where the agents have access to the set of all resources was studied in \cite{chun2004selfish}, where the authors were able to characterize the set of equilibrium points based on the parameters of the problem. However, unlike the uncapacitated case, there is no comparable characterization of equilibrium points in capacitated selfish replication games. In fact, when the agents have limited capacity, the situation could be much more complicated as the constraint couples the actions of agents much more than in the uncapacitated case or in replication games with servers.

Typically, capacitated selfish replication games are defined in terms of a set of available resources for each player, where the players are allowed to communicate through an undirected communication graph. Such a communication graph identifies the access cost among the players, and the goal for each player is to satisfy his/her customers' needs with minimum cost. Ideally, and in order to avoid any additional cost, each player only wants to use his/her own set of resources. However, due to limitation on capacity, players do not have access to all the resources and hence, they incur some cost by traveling over the network and borrowing some of the resources which are not available in their own caches from others in order to meet their customers' demands. The problem of finding an equilibrium for capacitated selfish replication games in the case of hierarchical networks was studied in \cite{gopalakrishnan2012cache}. Moreover, the class of capacitated selfish replication games with binary preferences has been studied in \cite{gopalakrishnan2012cache,etesami2014pure}, where ``binary preferences" captures the behavioral pattern where players are
equally interested in some objects.

In this paper, we consider the capacitated selfish replication game with binary preferences. In this model players act myopically and selfishly, while they are required to fully satisfy their customers' needs. Note that, although the players act in a selfish manner with respect to others in satisfying their customer needs, their actions are closely coupled with the others' and they do not have absolute freedom in the selection of their actions. It was shown in \cite{gopalakrishnan2012cache} that when the number of resources is 2, there exists a polynomial time algorithm $\mathcal{O}(n^3)$ to find an equilibrium. This result has been improved in \cite{etesami2014pure} to a linear time algorithm when the number of resources is bounded above by 5. However, in general there exist only exponential time algorithms for finding a pure-strategy Nash equilibrium. In this work we consider such games over general undirected networks and devise randomize algorithm based on random tree search to find the Nash equilibrium points of the system.

The paper is organized as follows. In Section~\ref{sec:game-model}, we introduce capacitated selfish replication games with binary preferences over general undirected networks. We review some salient properties of such games and include some relevant existing results on this problem. In Section \ref{sec:main-I}, we provide we characterzie the equilibrium points of the system as maximizers of a well-defined function and theroreticaly justify that randomization can be very beneficial for maximizing such function under certain cases. Following this, in Section \ref{sec:apx-alg} we device a random tree search algorithm for maximize the objective function and through numerous simulations we show that this method can be quite effective. We conclude the paper with identifying future directions of research in Section~\ref{sec:conclusion}.

\textbf{Notations}:
For a positive integer $n$, we let $[n]:=\{1,2,\ldots,n\}$. For a vector $v\in \R^n$, we let $v_i$ be the $i$th entry of $v$. We use $\mathcal{G}=([n], \mathcal{E})$ for an undirected underlying network with a node set $\{1,2,\ldots,n\}$ and an edge set $\mathcal{E}$. For any two nodes $i, j \in [n]$, we let $d_{\mathcal{G}}(i,j)$ be the graphical distance between them, that is, the length of a shortest path which connects $i$ and $j$. The diameter of a graph, denoted by $D$, is the maximum distance between any pair of vertices, that is, $D=\max_{i,j \in [n]} d_{\mathcal{G}}(i,j)$. Moreover, for an arbitrary node $i\in [n]$ and an integer $r\ge 0$, we define a ball of radius $r$ and center $i$ to be the set of all the nodes in the graph $\mathcal{G}$ whose graphical distance to the node $i$ is at most $r$, i.e., $B(i,r)=\{x\in \mathcal{V}| d_{\mathcal{G}}(i,x)\leq r\}$. We denote a specific Nash equilibrium by $P^*$. Finally, we use $|S|$ to denote the cardinality of a finite set $S$.
