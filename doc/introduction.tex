\section{Introduction}
With myriads of data from social science and emergence of large scale social and economic networks, studying the evolutionary behavior of complex decision systems has become a major issue in recent years. In fact, game theory is one of the strong mathematical tools which has proven quite useful in capturing and modeling the complex nature of human decision making in social and economics networks. Despite numerous advances and contributions made in this field of study, there are a lot of barriers and limitations concerning computational issues of different games such as finding Nash equilibrium points or even approximation the equilibrium points of the system.  
 

In this context, we consider in this paper a class of problems known as resource allocation problems. In general, resource allocation games are referred to a class of games in which a set of agents or players are competing for the same set of resources and has many applications in various areas such as load balancing, peer-to-peer systems, congestion games, web-caches, content management, and market sharing games among many others \cite{pollatos2008social,pacifici2012convergence,gopalakrishnan2012cache,goemans2006market,masucci2014strategic,ackermann2008impact,fabrikant2004complexity}. In fact, one of the important features of such games is to increase the reliability of the entire network with respect to customer needs and to improve the availability of the resources for users.            

Typically, resource allocations games can be divided in two general groups: \textit{uncapacitated} and \textit{capacitated}. In uncapacitated allocation games the agents can keep even up to all the possible resources in their caches. However, there is an associated cost of keeping certain resources in the catch and this eliminates the posibility that every agent keep all the available resources in its catch. An instance of such games has been studied in \cite{chun2004selfish}, where the authors were able to fully characterize the set of equilibrium points. However, in the capaciteted case the situation could be much more complicated as the constraints on the capacities of the agents couples the actions of agents much more than in the uncapacitated case. As a result there is no comparable characterization of equilibrium points in capacitated selfish replication games. 

<<<<<<< HEAD
Typically, capacitated selfish replication games are defined in terms of a set of available resources for each player, where the players are allowed to communicate through an undirected communication graph. Such a communication graph identifies the access cost among the players, and the goal for each player is to satisfy his/her customers' needs with minimum cost. Ideally, and in order to avoid any additional cost, each player only wants to use his/her own set of resources. However, due to limitation on capacity, players do not have access to all the resources and hence, they incur some cost by traveling over the network and borrowing some of the resources which are not available in their own caches from others in order to meet their customers' demands. The problem of finding an equilibrium for capacitated selfish replication games in the case of hierarchical networks was studied in \cite{gopalakrishnan2012cache}. Moreover, the class of capacitated selfish replication games with binary preferences has been studied in \cite{gopalakrishnan2012cache,etesami2014pure}, where "binary preferences" captures the behavioral pattern where players are equally interested in some objects.

In this paper, we consider the \emph{Capacitated Selfish Replication (CSR)} game with binary preferences. In this model players act myopically and selfishly, while they are required to fully satisfy their customers' needs. Note that, although the players act in a selfish manner with respect to others in satisfying their customer needs, their actions are closely coupled with the others' and they do not have absolute freedom in the selection of their actions. It was shown in \cite{gopalakrishnan2012cache} that when the number of resources is 2, there exists a polynomial time algorithm $\mathcal{O}(n^3)$ to find an equilibrium. This result has been improved in \cite{etesami2014pure} to a linear time algorithm when the number of resources is bounded above by 5. However, in general there exist only exponential time algorithms for finding a pure-strategy Nash equilibrium. In this work we consider such games over general undirected networks and devise randomize algorithm based on random tree search to find the Nash equilibrium points of the system.

%##################

We use random tree search algorithm to locate Nash equilibrium points in CSR game. The random tree search algorithm is an efficient randomized algorithm for searching the state space of games. The random tree algorithm works by sampling the allocation space of the CSR game. The algorithm searches for the optimum allocation, resulting in the Nash equilibrium.
The random tree search is guided by maximizing a cost function associated with each allocation. The random tree grows multiple tree walks from an initial allocation. The random tree branches the simulation walks to bias the walk toward the optimim allocation and avoiding the local minimas.
Recently, stochastic random tree-based search methods have emerged as an alternative to classic optimization algorithms.  The random tree algorithm has been used to optimize analog circuits \cite{adel2015}, robotic motion planning \cite{Lavalle2006}, and deep learning\cite{silver2016}. In comparison to the classic optimization algorithms such as simulated annealing or hill climbing, the random tree search is more efficient and does not get stuck in local minimas \cite{adel15, Lavalle2006}. 

%##################

The paper is organized as follows. In Section~\ref{sec:game-model}, we introduce capacitated selfish replication games with binary preferences over general undirected networks. We review some salient properties of such games and include some relevant existing results on this problem. In Section \ref{sec:main-I}, we provide we characterzie the equilibrium points of the system as maximizers of a well-defined function and theroreticaly justify that randomization can be very beneficial for maximizing such function under certain cases. Following this, in Section \ref{sec:apx-alg} we device a random tree search algorithm to maximize the objective function. We demonstrate, through numerous simulations, this method is very effective for finding the equilibrium points on large scale graphs in practical settings. We conclude the paper with identifying future directions of research in Section~\ref{sec:conclusion}.

\textbf{Notations}:
For a positive integer $n$, we let $[n]:=\{1,2,\ldots,n\}$. For a vector $v\in \R^n$, we let $v_i$ be the $i$th entry of $v$. We use $\mathcal{G}=([n], \mathcal{E})$ for an undirected underlying network with a node set $\{1,2,\ldots,n\}$ and an edge set $\mathcal{E}$. For any two nodes $i, j \in [n]$, we let $d_{\mathcal{G}}(i,j)$ be the graphical distance between them, that is, the length of a shortest path which connects $i$ and $j$. The diameter of a graph, denoted by $D$, is the maximum distance between any pair of vertices, that is, $D=\max_{i,j \in [n]} d_{\mathcal{G}}(i,j)$. Moreover, for an arbitrary node $i\in [n]$ and an integer $r\ge 0$, we define a ball of radius $r$ and center $i$ to be the set of all the nodes in the graph $\mathcal{G}$ whose graphical distance to the node $i$ is at most $r$, i.e., $B(i,r)=\{x\in \mathcal{V}| d_{\mathcal{G}}(i,x)\leq r\}$. We denote a specific Nash equilibrium by $P^*$. Finally, we use $|S|$ to denote the cardinality of a finite set $S$.
=======
In this paper we focus on an instance of capacitated resource allocation games with binary preferences as was introduced in \cite{gopalakrishnan2012cache}, where ``binary preferences" captures the behavioral pattern where players are equally interested in some objects. In the balance of this paper, our focus will be on such games, which for simplicity we refer to as CSR games. In fact, CSR games are defined in terms of a set of available resources for each player, where the players are allowed to communicate through an undirected communication graph. Such a communication graph identifies the access cost among the players, and the goal for each player is to satisfy his/her customers' needs with minimum cost. 

The problem of finding an equilibrium for CSR games have been studied in \cite{gopalakrishnan2012cache,etesami2014pure}. It was shown in \cite{gopalakrishnan2012cache} that when the number of resources is 2, there exists a polynomial time algorithm $\mathcal{O}(n^3)$ to find an equilibrium, where $n$ is the number of players in the game. This result has been improved in \cite{etesami2014pure} to a linear time algorithm $\mathcal{O}(n)$ when the number of resources is bounded above by 5. Moreover, a quasi-polynomial approximation algorithm $\mathcal{O}(n^{\ln D})$ for approximating any Nash equilibrium of the system within a constant factor has been given in \cite{etesami2015approximation}, where $D$ denotes the diameter of the network. In this paper we consider CSR games over general undirected networks and devise a randomized algorithm based on random tree search to find the Nash equilibrium points of the system.
 
The paper is organized as follows. In Section~\ref{sec:game-model}, we introduce CSR games over general undirected networks and review some relevant existing results on this problem. In Section \ref{sec:main-I}, we  characterize the equilibrium points of the system as maximizers of a well-defined function and theoretically justify that randomization can be quite beneficial for maximizing such function under certain cases. Following this, in Section \ref{sec:apx-alg} we device a random tree search algorithm for maximize the objective function. Through numerous simulation results we show that this method can be indeed quite effective. We conclude the paper with identifying future directions of research in Section~\ref{sec:conclusion}. 

\textbf{Notations}: 
For a positive integer $n$, we let $[n]:=\{1,2,\ldots,n\}$. We use $\mathcal{G}=([n], \mathcal{E})$ for an undirected underlying network with a node set $\{1,2,\ldots,n\}$ and an edge set $\mathcal{E}$. For any two nodes $i, j \in [n]$, we let $d_{\mathcal{G}}(i,j)$ be the graphical distance between them, that is, the length of a shortest path which connects $i$ and $j$. For an arbitrary node $i\in [n]$ and an integer $r\ge 0$, we define a ball of radius $r$ and center $i$ to be the set of all the nodes in the graph $\mathcal{G}$ whose graphical distance to the node $i$ is at most $r$, i.e., $B_{\mathcal{G}}(i,r)=\{x\in \mathcal{V}| d_{\mathcal{G}}(i,x)\leq r\}$. We denote the cardinality of a finite set $A$ by $|A|$.
>>>>>>> origin/master
